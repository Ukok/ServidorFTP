%%%%%%%%%%%%%%%%%%%%%%%%%%%%%%%%%%%%%%%%%%%%%%%%%%%%%%%%%%%%%%%%%%%%%%%%%%%%%%%%%%%%%%%%%%%%%%%%%%%%
\section{Configuraci\'on}

  \begin{frame}
    \frametitle{Configuraci\'on}
    \begin{itemize}
      \item Respaldo del archivo de vsftpd.conf
      \item Se crea un shell virtual o fantasma.
      \item Se agrega un grupo que controle al servicio de vsftpd.
    \end{itemize}
    \begin{shell}
      \cmd{ cp /etc/vsftpd.conf /etc/vsftpd.conf.backup }\\
      \cmd{ mkdir /bin/ftp }
      \cmd{ sh -c \tqd echo ''/bin/ftp'' >> /etc/shells\tqd}\\
      \cmd{ groupadd ftp }
      \cmd{ mkdir /home/ftp }
    \hline\end{shell}
  \end{frame}
%%%%%%%%%%%%%%%%%%%%%%%%%%%%%%%%%%%%%%%%%%%%%%%%%%%%%%%%%%%%%%%%%%%%%%%%%%%%%%%%%%%%%%%%%%%%%%%%%%%%

  \begin{frame}
    \frametitle{Configuraci\'on}
    \begin{itemize}
      \item Crear el home donde cada usuario va encontrar su informaci\'on.
      \item Se crea un directorio en el home de cada usuario, al cual se le dar\'an permisos, para 
      que pueda subir archivos o crear directorios mediante ftp.
      \item Se modifica los permisos a las carpetas y la propiedad de las mismas.
    \end{itemize}
    
    \begin{shell}
      \cmd{ mkdir /home/ftp/usuario }
      \cmd{ chmod 555 /home/ftp/usuario }\\
      \cmd{ mkdir /home/ftp/usuario/upload }
      \cmd{ chmod 0755 /home/ftp/usuario/upload }
    \hline\end{shell}
  \end{frame}
%%%%%%%%%%%%%%%%%%%%%%%%%%%%%%%%%%%%%%%%%%%%%%%%%%%%%%%%%%%%%%%%%%%%%%%%%%%%%%%%%%%%%%%%%%%%%%%%%%%%

   \begin{frame}
     \frametitle{Configuraci\'on}
    \begin{itemize}
      \item Se crea el nuevo usuario.
      \begin{itemize}
        \item Se agrega al grupo ftp.
        \item Se le asigna su directorio home.
        \item Se le asigna el shell virtual.
      \end{itemize}
      \item Se modifica la propiedad de las carpetas del home.
    \end{itemize}
    
    \begin{shell}
      \cmd{ useradd -g ftp -d /home/ftp/usuario -s /bin/ftp -c ''Nombre del usuario'' usuario }\\
      \cmd{ chown usuario.ftp /home/ftp/usuario/* -R }
    \hline\end{shell}
   \end{frame}
%%%%%%%%%%%%%%%%%%%%%%%%%%%%%%%%%%%%%%%%%%%%%%%%%%%%%%%%%%%%%%%%%%%%%%%%%%%%%%%%%%%%%%%%%%%%%%%%%%%%

   \begin{frame}
     \frametitle{Configuraci\'on}
     \framesubtitle{Archivo vsftpd.chroot\_list}
     \begin{itemize}
       \item Crear el archivo vsftpd.chroot\_list
       \item Agregar el usuario al archivo, tal como esta en /etc/passwd
     \end{itemize}
    \begin{shell}
      \cmd{ touch /etc/vsftpd.chroot\_list }
      \cmd{ sh -c \tqd cat /etc/passwd | grep usuario >> /etc/vsftpd.chroot\_list \tqd }
    \hline\end{shell}
   \end{frame}
%%%%%%%%%%%%%%%%%%%%%%%%%%%%%%%%%%%%%%%%%%%%%%%%%%%%%%%%%%%%%%%%%%%%%%%%%%%%%%%%%%%%%%%%%%%%%%%%%%%%

  \begin{frame}
    \frametitle{Configuraci\'on}
    \framesubtitle{Archivo vsftpd.conf}
    Se modifica algunas opciones en el archivo de configuraci\'on.
    \begin{shell}\\
      listen=YES\\
      local\_enable=YES\\
      write\_enable=YES\\
      xferlog\_file=/var/log/vsftpd.log\\
      ftpd\_banner=Welcome to Equipo-7 FTP service.\\ \\
    \hline\end{shell}
   \end{frame}
   
%%%%%%%%%%%%%%%%%%%%%%%%%%%%%%%%%%%%%%%%%%%%%%%%%%%%%%%%%%%%%%%%%%%%%%%%%%%%%%%%%%%%%%%%%%%%%%%%%%%%

    \begin{frame}
    \frametitle{Configuraci\'on}
    \framesubtitle{Archivo vsftpd.conf}
    Con estas configuraciones limitamos a cada usuario a su directorio personal.
    \begin{shell}\\
      chroot\_local\_user=YES\\
      chroot\_list\_enable=YES\\
      chroot\_list\_file=/etc/vsftpd.chroot\_list\\ \\
    \hline\end{shell}
   \end{frame}
   
%%%%%%%%%%%%%%%%%%%%%%%%%%%%%%%%%%%%%%%%%%%%%%%%%%%%%%%%%%%%%%%%%%%%%%%%%%%%%%%%%%%%%%%%%%%%%%%%%%%%

  \begin{frame}
    \frametitle{Configuraci\'on}
    \framesubtitle{FTPS}
    Se crea las llaves.
    
    \begin{shell}
      \cmd{openssl req -sha256 -x509 -nodes -days 365 -newkey rsa:4096 
      -keyout /etc/ssl/private/vsftpd.key -out /etc/ssl/certs/vsftpd.pem}\\
    \hline\end{shell}
    
    Se modifica algunas opciones en el archivo de configuraci\'on.
    
    \begin{shell}
      rsa\_cert\_file=/etc/ssl/certs/vsftpd.pem\\
      rsa\_private\_key\_file=/etc/ssl/private/vsftpd.key\\
      ssl\_enable=YES\\
    \hline\end{shell}
   \end{frame}
   
%%%%%%%%%%%%%%%%%%%%%%%%%%%%%%%%%%%%%%%%%%%%%%%%%%%%%%%%%%%%%%%%%%%%%%%%%%%%%%%%%%%%%%%%%%%%%%%%%%%%
